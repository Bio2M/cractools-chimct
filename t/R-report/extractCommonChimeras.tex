\section{SYNOPSIS\label{SYNOPSIS}\index{SYNOPSIS}}
\begin{verbatim}
    extractCommonChimeras.pl [-h] [--min-ranking <score>] [--threshold <min_redundancy>] [--summary <output_file>] [--output <output_file>]  [--Rdata <output_file>] <file1.csv> <file2.csv> ... <filen.csv>
    The mandatory arguments are at least two csv files generated by chimeraPipeline.
    --output: put the filename to write the results (STDOUT by default)
    --min-ranking: a minimal rank to consider the chimera (0 by default, ie no minimal rank)
    --threshold: the minimal redundancy to consider a redundant chimera, ie the min number of  
                 different samples where a same chimera is found (2 by default)
    --summary: make some statistics in the output_file if the argument is defined
    --Rdata: format the statistics for R graphics in the output_file if the argument is defined
\end{verbatim}
\section{OPTIONS\label{OPTIONS}\index{OPTIONS}}
\begin{verbatim}
    -h, --help
    --min-score=I<score>
    --summary=I<output_FILE>
    --threshold=I<min_redundancy>
    --Rdata=I<output_FILE>
\end{verbatim}
\section{OUTPUT FORMAT\label{OUTPUT_FORMAT}\index{OUTPUT FORMAT}}


This is a descrition of the fields of the output file format. Each line correspond to a redundant chimera
identified and annotated by the chimeraPipeline. Chimeras are ordered by Score, Class and number
of redundant chimeras.

\begin{enumerate}

\item \textbf{Id}                 - A Uniq Id for each chimera. This id is composed by \texttt{sample\_name:chimera id}.
\item \textbf{Name}               - Fusion genes names separated by three dashes ('---')
\item \textbf{Chr1}               - Chromosome of the 5' part of the chimera
\item \textbf{Pos1}               - Genomic positions of the 5' part of the chimera
\item \textbf{Strand1}            - Genomic strand of the 5' part of the chimera. If sample is not \texttt{--stranded}
                              No assumption can be made about the strand
\item \textbf{Chr2}               - Chromosome of the 3' part of the chimera. Same as \textit{Chr2}, unless it is a class 1 chimera (translocation).
\item \textbf{Pos2}               - Genomic positions of the 3' part of the chimera
\item \textbf{Strand2}            - Genomic strand of the 3' part of the chimera. If sample is not \texttt{--stranded}
                              No assumption can be made about the strand
\item \textbf{Average\_Rank}       - Average rank of all redundant chimeras. 
                              The rank is based on confidence about chimera's positivity (add more details)
\item \textbf{Class}              - Chimeric class from 1 to 4. (add more details)
\item \textbf{Redundancy}         - The redundancy for a given chimera correponds to the number of different sample where the chimera is found
\item \textbf{Listof (Sample,Spanning\_junction,Spanning\_PE,Rank)}      
                               - Sample identified the experiment
                                 Spanning junction read that contains the chimeric junction 
                                 Spanning paired-end reads that contains the chimeric junction but in the non-sequenced part
                                 Rank computed by the chimeraPipeline for the chimera in that Sample
\item \textbf{Comments}           - Several comments about the chimera and the rank computed\end{enumerate}
\section{DESCRIPTION\label{DESCRIPTION}\index{DESCRIPTION}}
\begin{verbatim}
    This script permets to group redondant chimeras between different samples and conditions.
\end{verbatim}
\section{REQUIRES\label{REQUIRES}\index{REQUIRES}}
\begin{verbatim}
    Perl5.
    Getopt::Long
    Pod::Usage
\end{verbatim}
\section{AUTHOR\label{AUTHOR}\index{AUTHOR}}
\begin{verbatim}
    Nicolas PHILIPPE <nicolas.philippe@inserm.fr>
\end{verbatim}
